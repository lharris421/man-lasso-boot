\section{Introduction}

Since its unveiling by \cite{Tibshirani1996}, the Lasso (Least Absolute Shrinkage and Selection Operator) has been a popular model of choice, even in cases which would not necessarily be classified as high-dimensional. Its ability to both do variable selection and estimation lends itself to a particular ease. Additionally, in scenarios where both predictive accuracy and interpretability are desired, the Lasso excels. This is particularly useful for models with a large number of covariates and the underlying model is believed to be sparse but where the exact subset of significant predictors is unknown. In such scenarios, the Lasso helps to identify the most relevant variables leading to a more parsimonious and hence interpretable model. \cite{HTF2009} provide an overview of the Lasso's performance in various settings, demonstrating its optimality in various examples in which the situations and results desired are similar to those previously mentioned. With that said, inference for the lasso has proven to be a difficult task, especially when one desires to obtain proper confidence intervals.

The challenges of constructing confidence intervals have lead to various alternative approaches in the literature. For example, the concept of post-selection inference as discussed in \cite{LeeEtAl2016}, which aims to account for the uncertainty in model selection by conditioning on the selected model or \cite{ZhangZhang2014} which focuses on a desparsified approach to correct the bias of the lasso and facilitate more traditional forms of inference. Some focus has been placed on using the bootstrap to produce confidence intervals, however, the conventional opinion since \cite{Chatterjee2010} is that the traditional use of bootstrapping to produce confidence intervals for penalized regression is suboptimal. The suboptimality arises directly due to the inherent bias and sparsity of estimates produced by soft thresholding.

Thus, we take a different perspective and raise a new question to conventional wisdom: does the bootstrap not work or do we need to rethink the properties of confidence intervals in high dimensional settings? We find that it's a combination of both. There does need to be a methodological fix, but also we need to rethink high dimensional confidence intervals (HDCIs).

Section~\ref{Sec:Difficulties} ...

\section{Difficulties of Confidence Intervals for lasso}\label{Sec:Difficulties}

We see the difficulty of constructing HDCIs arising from two problems when proposing to use bootstrapping for penalized regression, we will refer to these obstacles as,

\begin{enumerate}
\item the Bias Tradeoff, and
\item the Epsilon Conundrum.
\end{enumerate}

The Bias Tradeoff refers to the inherent bias introduced by penalization in order to be able to fit over-saturated models. This penalization leads to coverages higher than nominal coverage rates for coefficients with true values at or very near zero and leads to lower coverage when coefficients are non-zero. We argue that coverage in the traditional sense is too ridged of a paradigm to apply to penalized regression inference. Instead, we offer a different perspective and argue that the impact that penalization induced bias has is an inherent feature of penalized regression rather than a flaw. Since high dimensional problems often necessitate such an alternate approach, we offer guidance on how to interpret the confidence intervals for lasso as presented in this paper. Additionally, we compare the proposed method to two other inferential methods which take a different perspective.

Before going further, it will be helpful to specify the behavior of the intervals we are interested in. Specifically, we are interest in intervals that are faithful to the model being fit. This inherently excludes most intervals that attempt to debias. This restriction imposes a behavior on the intervals. Since, bias being introduced into the model has varying effects on coefficients of increasing magitude, this implies that interval behavior will vary with the magnitude of $\beta$.

By limiting focus to intervals that are true to the model fit, we have little control over the center of the intervals. As the magnitude of $\beta$ increases, so generally does the bias. As we will see, the width of proposed intervals also generally does increase, but at a relatively lower rate than the bias. As a direct result, we would expect a ``faithful'' confidence interval produce different coverage rates as a function of the magnitude of $\beta$.

Instead, we propose a different focus: overall coverage. As a direct result of this, one must expect that such intervals are going to over cover values of $\beta$ that are small and undercover those which are larger in magnitude. The goal then being finding a method that is well calibrated to provide overall coverage near that of nominal.

The second problem, the Epsilon Conundrum, is also related to the shrinkage but arises as an arguably more disturbing manifestation: confidence intervals of length zero or, more often, with a single endpoint that is exactly zero. Some penalized regression methods, particularly the lasso, often result in a sparse solution. If using a traditional quantile based bootstrap confidence interval, this will lead to an interval of [0,0] if a given variable is rarely or never included in the active set. As the dimensionality of the problem grows, this becomes an increasing occurrence leading to a large majority of intervals possessing a length of zero. When the true value of the coefficient is 0, the interval at least contains the truth. However, this issue is particularly troublesome when one considers what happens when the true value is not precisely zero, as is likely the case in most reasonable scenarios. By just shifting the true value by $\eps$, an immediate drop in coverage would occur, hence the name: the Epsilon Conundrum.

\begin{figure}[hbtp]
  \includegraphics[width=\linewidth]{traditional}
  \caption{\label{Fig:traditional} Caption goes here}
\end{figure}

The later of these two problems is addressed in our novel approach to producing bootstrap based confidence intervals for the lasso.

\section{Notation}

Throughout this paper, certain notation will be repeatedly used. $B$ will represent the total number of bootstrap datasets generated. $\boldsymbol{X}^b$ and $\boldsymbol{y}^b$ refer to the $b^{th}$ bootstrap sample. Similarly, $\hat{\beta}^b_j$ will refer to the estimate for $\beta_j$ from the lasso fit to $\boldsymbol{X}^b$ and $\boldsymbol{y}^b$. However, the previous notion with a superscript $b$ will often be used in a nonspecific manner to indicate that an estimate is with respect to a bootstrap draw rather than a specific bootstrap draw.

\section{Lasso Bootstrap Confidence Intervals}

\subsection{An Overview of Proposed Methods}

In what follows, we examine four alternative methods for obtaining bootstrap draws.

The first is the traditional bootstrap which simply involves taking the point estimate for each $\beta_j$ for each respective bootstrap sample. As mentioned previously, and as we will see, this has clear pitfalls which the subsequent three alternatives attempt to address.

The second is the posterior bootstrap. This leverages the Bayesian formulation of the lasso. For each bootstrap sample, a random draw from the full conditional posterior is obtained instead of taking the point estimate (equivalent to the mode) of the posterior.

The posterior bootstrap tends to be too aggressive in the additional variability introduced. As such, the third alternative borrows behavior from both the traditional and the posterior bootstrap and is appropriately called the hybrid bootstrap in this manuscript. The hybrid bootstrap only samples from the full conditional if the point estimate for a given $\beta_j$ is equal to zero for that bootstrap sample. This alternative will be the main focus of the manuscript.

The fourth and final method is a naive approach at producing debiased intervals and as such is referred to as the debiased bootstrap in this manuscript. Instead of taking the point estimate as the traditional method would, this method uses the correlation between the partial residuals obtained from the lasso with $\boldsymbol{x}_j$. As such, one might expect this to address both issues mentioned above. As we will see, it does in fact, however, its performance in general is suboptimal when $p < n$.

\subsection{Intuition behind the hybrid bootstrap}

Given that the hybrid bootstrap is the main focus of this manuscript, it will be worth while to briefly consider the intuition driving this proposed sampling method. Recall, potentially the most striking issue with the traditional bootstrap is that it can produce intervals which are singleton at 0. This occurs when $\hat{\beta}_j^b = 0$ for at least $B * (1 - \alpha)$ draws. To address this issue, then, adjustment is only needed when $\hat{\beta}_j^{b} = 0$. Next, we consider the behavior of this interval from a logical perspective.

For a moment, hold p constant. As n increases and we let $\lambda$ get selected by CV, this method will converge to that of applying the traditional bootstrap to a classical linear model. This is the case because with increasing n, $\lambda_{CV} \rightarrow 0$. On the other hand, as n decreases $\lambda_{CV} \rightarrow \lambda_{max}$ and $\hat{\boldsymbol{\beta}}\rightarrow \boldsymbol{0}$. On this end of the extreme, bootstrap samples are then drawn from the marginal posterior. The coverage behavior is well understood for the bootstrap applied to linear regression as n increases. By only sampling from the posterior when $\hat{\beta}_j^b = 0$, we minimize unnecessary variability as n increases. The behavior on the other end of the extreme is relatively less understood compared to the asymptotics of the traditional bootstrap. However, we argue that leveraging the posterior lends itself to reasonable behavior especially when the alternative would be confidence intervals which are all identical to zero. Such intervals, one could argue, would necessarily have 0\% coverage. Conversely, as we will see, with smaller values of n, the posterior tends to lead us to wider confidence intervals. See Section~\ref{Sec:full-cond}.

\subsection{Conceptual Implementation}
\label{Sec:full-cond}

In this section, we describe the details and important considerations for the sampling performed in the posterior and the hybrid bootstrap.

As with other penalized regression models, the lasso can be formulated as a Bayesian regression model by setting an appropriate prior. This was addressed initially by \cite{Tibshirani1996} and covered more extensively by \cite{Park2008}. For the lasso, the prior is a Laplace distribution, also referred to as the double-exponential distribution:

\as{\p(\bb) = \prod_{j = 1}^{p} \frac{\gamma}{2}\exp(-\gamma \abs{\beta_j}), \gamma > 0}

With this prior, the lasso estimate $\bbh(\lam)$ is the posterior mode of $\bb$ when $\lam = \gamma \frac{\sigma^2}{n}$. We purpose leveraging this relationship for building confidence intervals.

Unfortunately, a Normal likelihood and Laplace prior are not conjugate and in general, the absolute value in the exponent of the Laplace makes many common manipulations for the posterior more difficult. Luckily, however, the full conditional posterior can be shown to be a mixture of a right and left truncated normal where the truncation occurs at zero for right and left tails respectively. To obtain a full conditional posterior, we use the partial residuals, $\r_{-j}$, in the likelihood. This is natural when considering the common CD algorithms used to arrive at lasso estimates. In this manuscript, we will assume that $\X$ has been standardized s.t. $\x_j^T\x_j = n$.

Note that in what follows, the full conditional represents the distribution for a given value of $\lambda$ and $\sigma^2$. Further discussion will be given to selecting these values later, however, in the meantime to reduce notational distractions we will treat them as known quantities and implicitly condition on them.

Then, for $\beta_j$,

\as{
L(\beta_j | \r_{-j}) &\propto \exp(-\frac{n}{2\sigma^2} (\beta_j^2 - 2z_j\beta_j)), \\
\text{ where } z_{j} &= \frac{1}{n} \x_{j}^{T}\r_{-j} \text{, and} \\
P(\beta_j | \r_{-j}) &\propto
\begin{cases}
C_{-} \exp(-\frac{n}{2\sigma^2} (\beta_j - (z_j + \lambda))^2), \text{ if } \beta_j < 0, \\
C_{+} \exp(-\frac{n}{2\sigma^2} (\beta_j - (z_j - \lambda))^2), \text{ if } \beta_j \geq 0 \\
\end{cases}
}

Where, $C_{-} = \exp(\frac{z_j \lambda n}{\sigma^2})$ and $C_{+} = \exp(-\frac{z_j \lambda n}{\sigma^2})$.

What makes this formulation attractive is that a mapping allows tail probabilities from the posterior to be translated to probabilities onto corresponding known normal distributions (i.e. $\N(z_j \pm \lambda, \frac{\sigma^2}{n})$). The allows for numerically stable and efficient sampling to be obtained from the full conditional posteriors. Details of how sampling is performed are outside of the scope of this manuscript and are provided in Supplement~\ref{Sup:A}. For the remainder of the manuscript, we simply refer to this process as ``sampling from the full conditional.''

Again, recall that this solution corresponds to a particular value of $\lam$ and $\hat{\sigma}^2$. Our recommendation is to use CV to select $\lam$ and produce an estimate for $\sigma^2$ and produce bootstrap confidence intervals corresponding to these values. Performance under this recommendation will be focused on in the Results section. However, it should be noted that producing an estimate for $\sigma^2$ in this manner implicitly depends on $\lam$, so new estimates for $\sigma^2$ should be obtained for each value of $\lam$. When referring to $\hat{\sigma}^2$ it will always be as a function of $\lam$ which we drop for notational convenience.

All together this leads to the following steps to obtain CIs through the various alternative bootstraps:

\begin{enumerate}
\item Perform CV using the original data to select $\lam$ and estimate $\hat{\sigma}^2$
\item For b $\in \lbrace 1, \ldots, B \rbrace$:
\begin{enumerate}
\item Obtain a pairs bootstrap sample, $\X^b$ and $\y^b$
\item Fit lasso with $\X^b$ and $\y^b$, obtain $\bbh^b$, $\boldsymbol{z}^b$
\item For j $\in \lbrace 1, \ldots, p \rbrace$:
	\begin{algorithmic}
	\Switch{method}
    \Case{traditional}
      \Assert{$q_j = \bh_j$}
    \EndCase
	  \Case{posterior}
    \Assert{
      $p^* = \Unif(0, 1)$ \\
      \hspace{1.95cm} $q_j = P^{-1}(p^*|\r_{-j})$
    }
    \EndCase
    \Case{hybrid}
      \Assertif{$\bh_j \neq 0$}{$q_j = \bh_j$}
      \Assertelse{
        $p^* = \Unif(0, 1)$ \\
        \hspace{1.8cm} $q_j = P^{-1}(p^*|\r_{-j})$
      }
    \EndCase
    \Case{debiased}
	    \Assert{$q_j = z_j$}
    \EndCase
	\EndSwitch 
	\end{algorithmic}
\item Save $\q$
\end{enumerate}
\item Row bind all B $\q$ vectors to obtain a $B \times p$ matrix of bootstrap draws
\item For each $\beta_j$, compute the quantiles for $p_L = (.5 - \l/2)$ and $p_U = (.5 + \l/2)$ from the $j^{th}$ column of the draws to produce a final confidence interval estimate for significance level $\l$.
\end{enumerate}

\section{Results}

The results start off by considering interval behavior before moving onto a number of other scenarios which shed light on the robustness of the Hybrid bootstrap method. We then move onto a brief comparison of the Hybrid bootstrap intervals to those produced by Ridge regression in a case where two variables are highly correlated. Comparison of Hybrid to other methods continues with a look at two other HDCI methods before rounding out results with an application to two datasets in which Hybrid is again compared to the two alternative HDCI methods.

Unless otherwise noted, the nominal coverage rate used in this manuscript is 80\%.

\subsection{Thm}

Theorem: If the likelihood is correctly specified according to the true data generating mechanism, $p(\boldsymbol{y} | \bt)$, then credible sets obtained from $p(\bt|\boldsymbol{y})$ will have proper coverage when averaged over the prior distribution $p(\bt)$... \logan{Still unsure how to make the statement that the prior matches the true distribution without sounding too Bayesian.} \\

Proof: By definition, a $100(1-\alpha\%)$ credible region for $\boldsymbol{\theta}$ is any set $\boldsymbol{A}$ s.t. $p(\bt \in \boldsymbol{A} | \boldsymbol{y}) \geq 1 - \alpha$. For a given $\bt_0$, the coverage probability can be defined as $\int I(\bt_0 \in \boldsymbol{A} | \boldsymbol{y}) p(\boldsymbol{y} | \bt_0)d\boldsymbol{y}$. Averaged over $p(\bt)$, the average coverage probability can be defined as:

\as{
  \begin{aligned}
  \int \int I(\bt \in \boldsymbol{A} | \boldsymbol{y}) p(\boldsymbol{y} | \bt)d\boldsymbol{y}p(\bt)d\bt &= \int \int I(\bt \in \boldsymbol{A} | \boldsymbol{y}) p(\boldsymbol{y} | \bt)p(\bt)d\boldsymbol{y}d\bt \\
  &=  \int \int I(\bt \in \boldsymbol{A} | \boldsymbol{y}) p(\bt|\boldsymbol{y})p(\boldsymbol{y})d\boldsymbol{y}d\bt\footnote{This step is only valid if the assumptions are met.} \\
  &=  \int \int I(\bt \in \boldsymbol{A} | \boldsymbol{y}) p(\bt|\boldsymbol{y})d\bt p(\boldsymbol{y})d\boldsymbol{y} \\
  &=  \int \int_{\boldsymbol{A}} p(\bt|\boldsymbol{y})d\bt p(\boldsymbol{y})d\boldsymbol{y} \\
  &\geq (1-\alpha)  \int p(\boldsymbol{y})d\boldsymbol{y} \\
  &= 1-\alpha
  \end{aligned}
}

\logan{Need to fix footnote.}

\subsection{Interval Behavior}

We compare the proposed CI methods by showing the coverage of each, both overall and as the magnitude of $\beta$ changes, then present results on the width and bias of the intervals as a way of gaining insight into why the methods result in over or under coverage at various magnitudes of $\beta$.

\subsubsection{Coverage}\label{Sec:Coverage}

Before describing the simulation set up, we make the assertion that the ideal scenario for the proposed methods (Hybrid and Posterior) is when the true distribution of $\bb$ follows a laplace (double exponential) distribution. As such, the first simulation under consideration uses $\bb$s generated independently under a Laplace distribution. Note, the assertion that this data generating mechanism is ideal is based on Theorem~\logan{5.1}, \logan{update after wording set for thm}. This may raise the question of what rate is ideal. In fact, this question could be generalized for any of the distributions which $\bb$ is drawn from in the simulations that follow. However, the data generating mechanism used renders the choice of scale arbitrary. In what follows, we set SNR = 1 and $\sigma^2 = 1$ which imposes the restriction that $\bb^T\bb$ also equals 1. As a result, regardless of the scale parameter used, after standardization, the draws for $\bb$ will be identical (assuming the same seed is used).

Figure~\ref{Fig:laplace} displays simulation results for CIs obtained via the four previously described bootstrap methods. In this simulation, 100 independent datasets were generated and each bootstrap method was applied using $B = 1000$ bootstrap iterations. Each dataset was simulated as follows. $\X$ was generated independently with $n = p = 100$ and $\bb$ was generated from a Laplace distribution and standardized so that SNR = 1. Then, $\Y$ was constructed as $\Y = \X\bb + \boldsymbol{\epsilon}$, where $\boldsymbol{\epsilon} \overset{i.i.d}{\sim} N(0, 1)$. In Figure~\ref{Fig:laplace}, the dotted lines represents the average coverage for each method across all variables for all 100 datasets. The solid lines are estimates of coverage as a smooth function of $|\beta|$.

\begin{figure}[hbtp]
  \includegraphics[width=\linewidth]{laplace}
  \caption{\label{Fig:laplace} Results are from the simulation described in Section~\ref{Sec:Coverage}. The fitted curves are from Binomial GAMMs (one for each method) fit with coverage being modeled as a smooth function of $|\beta|$ and with a random intercept on dataset to account for deviations in coverage specific to a given randomly generated dataset. The data used for modeling contains a row for each variable per simulated dataset. The dashed lines represent the average for each method across all 100 independently generated datasets and the solid black line indicates the nominal coverage rate. The shaded distribution in the background depicts the positive tail of the distribution the $\beta$s were drawn from, $dexp(\tau = 2)$.}
\end{figure}

Before describing the observed coverage behavior, we draw brief attention to Figure~\ref{Fig:laplace} as a depiction that Hybrid is a mixture of Posterior and Traditional, being more like Posterior when $\beta$ is small in magnitude and increasingly more like Traditional as $\beta$ increases in magnitude.

Figure~\ref{Fig:laplace} shows that, as is generally the case when applied to the lasso, the Traditional bootstrap coverage is far below that of nominal. At the other end of the extreme, Posterior, which draws from the full conditional for every bootstrap draw, tends to produce intervals that significantly over cover relative to nominal coverage. Only Hybrid, which only samples from the full conditional when $\bh = 0$, and Debiased produce coverage near the $80\%$ nominal coverage rate.

To understand the coverage behaviors further, it helps to consider what happens near zero and far-from-zero. Traditional has low coverage both near and far-from zero. Debiased also under covers both near and far-from zero but to a much lesser extent and, of the methods considered here, generally sticks closest to nominal coverage regardless of the magnitude of $\beta$. Conversely, Posterior and Hybrid both display a significant degree of over coverage for values of $\beta$ near zero. However, like the other two methods, Posterior and Hybrid also both under cover far-from-zero, but to varying degrees. 

Put more explicitly, what is novel for Posterior and Hybrid is the decreasing trend with coverage rates very high for values of $\beta$ near zero and lower for values of $\beta$ larger in magnitude. In classical settings, we would expect coverage to be constant despite the value of $\beta$, which is not the case here. This decreasing pattern is typical for these methods and is to be expected from intervals arising from procedures that are faithful to the lasso model fit. As emphasized in Section~\ref{Sec:Difficulties}, introducing bias through penalization is an inherent feature of the lasso. This bias can largely explain the pattern observed. In fact, as we will see, the effect of penalization on the width and bias of the intervals can explain the coverage patterns seen for all four methods, which is where we now turn our attention to.

\subsubsection{Width and Bias}\label{Sec:Width and Bias}

\begin{figure}[hbtp]
  \includegraphics[width=\linewidth]{laplace_width_bias}
  \caption{\label{Fig:laplace_width_bias} Results are from the simulation described in Section~\ref{Sec:Coverage}. The fitted curves are from GAMMs (one for each method) fit with width and central bias being modeled as a smooth function of $|\beta|$. The data used for modeling contains a row for each variable per simulated dataset. Additionally, the models contain a random intercept on dataset to account for deviations in width and central bias specific to a given randomly generated dataset. Central bias is defined here as the difference between the center of an interval and the true value of $\beta$, with positive values indicating bias towards zero. Further discussion on central bias can be found in Section~\ref{Sec:Width and Bias}}
\end{figure}

Figure~\ref{Fig:laplace_width_bias} can be used to largely explain the coverage behavior seen in Figure~\ref{Fig:laplace}. The fitted curves in each plot are constructed in the same manner as for Figure~\ref{Fig:laplace}, albeit on the respective feature of interest. The left side of Figure~\ref{Fig:laplace_width_bias} shows that the interval width tends to increase as $\beta$ increases in magnitude. This behavior is related to the proportion of times a variable is selected to be in the model. When a given $\beta$ is near zero, its estimate will often be shrunk to zero which results in less variable bootstrap draws than for a variable which is always selected. Worded differently, a variable with a larger corresponding true value of $\beta$ will have a wider range of plausible estimates for each bootstrap draw than for a variable with a true value of $\beta$ nearer to zero, all else equal. This effect is most notable for the Traditional bootstrap, as it is the only method which will result in a draw exactly equal to zero when $\bh^b_j = 0$. The three other methods all introduce some additional variability in this scenario. This explains why Traditional produces much narrower intervals, especially for values of $\beta$ smaller in magnitude, and largely explains the drastic under coverage observed in the previous section. This is a manifestation of the Epsilon Conundrum as Traditional has a tendency to produce the interval $[0,0]$ for $\beta$s near zero. The plot of widths also helps display the underlying mechanism of over coverage observed for Posterior. We can see that compared to the other methods, the intervals for Posterior are significantly wider, regardless of the magnitude of $\beta$. The resulting over coverage suggests then that the intervals are too wide.

The right side of Figure~\ref{Fig:laplace_width_bias} provides an initial depiction of the bias of the confidence intervals produced by each method. Here, a positive value indicates bias towards zero. Before going forward, it is important to emphasize that, while informative, representation of interval bias has its limitations. While a clear definition exists for the bias of a point estimate, the bias of an interval lacks an accepted definition. This is for good reason, it is unclear what is meant by the bias of an interval. However, given the construction of the proposed intervals, the median bootstrap draw seems to be a logical definition of center and consequently what we use here for the determination of bias. With this point in mind, we see the expected behavior that center of the intervals tend to be increasingly biased towards zero for larger absolute values of $\beta$. This plot also indicates that Traditional, Posterior, and Hybrid have similar amounts of central bias, with Traditional having just a touch more. Although the Debiased method cuts the bias in about half relative to the other methods (but still shows an increasing trend), we see that it is unable to completely eliminate bias for reasons that are beyond the scope of this manuscript.

At this point, we would agree with any reader who feels that the preceding depiction of bias is incomplete. Indeed, there are intricacies in the coverages that are not captured in Figure~\ref{Fig:laplace_width_bias}. Accordingly, Figure~\ref{Fig:laplace_bias_nfb} provides additional information about the interval behavior, all of which could be considered under the umbrella of bias. Since there is no one great measure of bias, considering these multiple measures provides a reasonable way to communicate the bias related behavior more wholistically. The plots in Figure~\ref{Fig:laplace_bias_nfb} are constructed the same way as Figures \ref{Fig:laplace} and \ref{Fig:laplace_width_bias}. The upper left and right plots depict the probability that an interval misses the truth either towards or away from zero, respectively, as a function of the magnitude of $\beta$. The bottom left plot shows the difference between the probability a method misses towards zero and the probability it misses away. For the most part, as a result of bias, we see that the intervals miss towards zero which accordingly increases as the magnitude of $\beta$ increases. Note that although Hybrid does tend to miss towards zero more than Posterior that this is mostly a result of decreased width rather than increased central bias, which we saw in Figure~\ref{Fig:laplace_width_bias} is very comparable between the two methods. This set of plots indicates some misbehavior for Debiased, especially for $\beta$s near zero, where Debiased has a tendency to miss away from zero to quite a large degree. Although Hybrid also displays this tendency, it does so to an extent an order of magnitude less and quickly drops to negligible levels as the magnitude of $\beta$ increases. Alternatively, Debiased's tendency to miss away from zero is large enough that the difference observed in the lower left plot is negative for smaller values of $\beta$. What this indicates is that Debiased has a relative tendency to ``overshoot'' when $\beta$ is near zero (i.e. over correct). However, for moderately small values of $\beta$, the opposite becomes true with Debiased actually having a higher tendency to miss towards zero compared to Hybrid and Posterior. It isn't until $\beta$ is larger in magnitude that it appears coverage is truly favorable. With Debiased having misbehavior in both directions, the average bias in Figure~\ref{Fig:laplace_width_bias} is misleading, making Debiased look better than it should since these two suboptimal behaviors are allowed to balance each other out. However, this is not even the whole story. As set up, the probability that a method misses towards zero by definition converges to zero as $\beta \rightarrow 0$ (unless the method can produce bounds that are exactly equal to zero). This is because we defined a miss towards zero with the requirement that at least part of the interval lies between the truth and zero, with an interval with both bounds equal to zero always being a miss towards zero (unless the true value was zero, which occurs wp 0 in this simulation). If the entire interval was of a different sign than that of the true value (unless both signs were 0), we defined this as the dreaded Type 3 error, which is presented in the bottom right plot. Ignoring Traditional, again we see Debiased as a stand out, having a significant tendency for Type 3 errors, especially near zero. Hybrid again displays this tendency, but as with misses away from zero, to a much lesser extent.

\logan{Not that we need to give it much attention, but the little blip near zero for hybrid in the miss away and type 3 error (noting that this doesn't occur for posterior) is interesting.}

\begin{figure}[hbtp]
  \includegraphics[width=\linewidth]{laplace_bias_nfb}
  \caption{\label{Fig:laplace_bias_nfb} Caption goes here.}
\end{figure}

\logan{Make sure there isn't too much repeated unnecessarily between the preceding and next paragraphs.}

Now that we have a deeper understanding of the intervals' behaviors, we can use them to explain the trends in coverages seen in Figure~\ref{Fig:laplace}. Traditional has the most amount of bias, notably towards zero regardless of the magnitude of $\beta$, and produces narrow intervals. As a result, the Traditional bootstrap under covers regardless of the magnitude of $\beta$. On the opposite end of the extreme, the Posterior bootstrap tends to produce intervals that over cover, mainly due to having intervals that are in general too wide. However, the over coverage does not occur across the range of $\beta$ values. In this sense, the Posterior and the Hybrid share a common pattern. This pattern is explainable through the effect of bias towards zero and the respective widths of the intervals. For values of $\beta$ near zero, the effect of bias is minimal while the width is only minorly decreased leading to coverage levels near 1. However, as $|\beta|$, and the associated bias, increase, this eventually leads to lower coverage for values of $\beta$ larger in magnitude, as width does not increase accordingly (nor would we want it to). Although they share the same coverage pattern, the main difference between Hybrid and Posterior is that Hybrid produces narrower intervals, as it only samples from the Full Conditional Posterior (the driver of the increased width for Posterior) when $\bh = 0$. This leads to rates of coverage nearer to that of nominal. Another observable effect of this sampling mechanism is the behavior of Hybrid coinciding with Posterior and Traditional respectively. When $|\beta| < 0.1$, the behavior of Hybrid is similar to that of Posterior. However, Hybrid largely coincides with Traditional for $|\beta| > 0.2$, because at this point, $\beta$s this large rarely if ever have draws with $\bh = 0$. Between 0.1 and 0.2, Hybrid undergoes a transition period between Posterior and Traditional. Of the four methods considered, Debiased exhibits the most complex behavior. It has intervals similar in width to Hybrid, but nearly half the average bias for any given value of $\beta$, yet, it produces coverage below that of nominal but with rather consistent coverage despite the magnitude of $\beta$. Why is this the case? Well, we saw in Figure~\ref{Fig:laplace_bias_nfb} that it is due to the rather sporadic behavior of the Debiased intervals, especially for values of $\beta$ near zero. 

We hope the reader is exceedingly convinced that the Traditional bootstrap is a poor choice. Additionally, the posterior isn't bad, it just generally provides intervals significantly wider than that of the Hybrid bootstrap. As will be demonstrated throughout the remainder of this manuscript, when the Hybrid departs from nominal coverage, it generally does so in the direction of over-covering. Thus, in these scenarios, the wider intervals of the Posterior are strictly worse than those produced by the Hybrid. The remaining two methods, Hybrid and Debiased, both produce coverage levels that are reasonably near that of nominal. As such, we will give both further consideration. As just noted, for the simple case considered in the first simulation, Debiased does tend to produce coverage near nominal. However, Debiased, approaches nominal (with increasing sample size) from below as the uncorrected bias for larger magnitude $\beta$s slowly diminishes. Additionally we saw that even if overall coverage looks good, that the underlying behavior is less than ideal (especially the Type 3 error). Hybrid, on the other hand, behaves more stably and tends to approach nominal from above rather than below. This stable behavior extends to more complex data generating mechanisms, something that we will see does not hold true for Debiased in Section~\ref{Sec:Robustness}.

\subsection{Robustness}\label{Sec:Robustness}

This section explores a number of scenarios to help understand the behavior of the Hybrid bootstrap. It begins with a look at the coverage behavior when there is correlation among the predictors and provides a comparison to the behavior of Debiased. The results in this section also give the first depiction of how the behavior changes with increasing sample size. As mentioned previously, and as we will see in Section~\ref{Sec:Correlation}, Debiased begins to break down under this more complex scenario and as such this will conclude any further consideration of the Debiased method in this manuscript. Next, we consider how Hybrid performs under various distributions of $\beta$ and follow this up with a brief revisit to the Epsilon Conundrum. This section is rounded out with a look at how the coverage behavior changes with changes in the $\lambda$ used before finishing off with a demonstration that the behavior is consistent across different choices of nominal coverage.

\subsubsection{Correlation}\label{Sec:Correlation}

Now, we consider the behavior of both Debiased and Hybrid where the covariates are generated under increasing levels of autoregressive correlation. Other than the addition of correlation, the set up of the simulation for the results displayed in Figure~\ref{Fig:correlation_structure} is the same as for the first simulation described. The boxplots provide the coverages across the 100 simulated datasets. The results for Hybrid and Debiased are in the top and bottom row respectively with the amount of correlation starting at $\rho = 0.4$ in the first column and increasing to $0.6$ in the second column and $0.8$ in the third. 

As previously alluded to, even at the lower end of correlations introduced, Debiased noticeably under covers even when $n = 4p = 400$. This is only exacerbated with increasing correlation. This, along with previously observed behavior, points to a lack of robustness for the Debiased bootstrap method. Hybrid, however, fares better in the presence of correlation. It takes the highest amount of autoregressive correlation ($\rho = 0.8$) to produce coverage that is noticeably below nominal coverage. For a moderate amount of correlation ($\rho = 0.4$), there is little impact on the coverage of the intervals (see Figure~\ref{Fig:laplace_comparison} for $\rho = 0$ as a comparison). 

\begin{figure}[hbtp]
  \includegraphics[width=\linewidth]{correlation_structure}
  \caption{\label{Fig:correlation_structure} Caption goes here}
\end{figure}

Debiased, which admittedly was a naive attempt, has accumulated a number of drawbacks which bring in to question its stability and suggest it should be dropped from consideration along with Traditional and Posterior methods. The remainder of the manuscript focuses solely on the Hybrid bootstrap. This hints at an idea that will be explored further in Section~\ref{Sec:Comparison}, debasing comes at a cost. This cost manifests differently depending on the implementation, but there is one. In Section~\ref{Sec:Comparison}, comparison of Hybrid is done to two alternative methods, both of which attempt to provide constant coverage regardless of the magnitude of $\beta$. Although the cost for one is far more noticeable than it is for the other, both have more variable coverage rates and wider intervals. Often, bias is spoken about in a hushed tone, but we want to emphasize that bias isn't always a bad thing. 

\subsubsection{Distribution of Beta}\label{Sec:Distribution}

Now that we have seen that the Hybrid Bootstrap does relatively well in the presence of correlation, we turn our attention to performance under various distributions of $\beta$. Given that we have previously observed that the coverage for a given $\beta$ depends on its magnitude, it may be expected that the coverage will vary wildly depending on how $\bb$ is distributed. Table~\ref{Tab:dist_beta} shows the results of $\bb$ generated under 7 alternative distributions. The simulation set up was again the same as the initial simulation discussed, but just with the respective generating mechanism for $\bb$. While it is certainly the case that the distribution affects the coverage, the underlying patterns remain the same. Additionally, while distributions with most of their mass near zero result in over coverage, the opposite behavior is not observed. That is, even when most of the density is concentrated away from zero, as with the Beta distribution, the desired coverage properties remain intact.

\begin{table}[hbtp]
  \centering
  
\begin{tabular}[t]{>{}cccccc}
\toprule
\multicolumn{2}{c}{  } & \multicolumn{4}{c}{Sample Size} \\
\cmidrule(l{3pt}r{3pt}){3-6}
  & Distribution & 50 & 100 & 400 & 1000\\
\midrule
\includegraphics[width=0.67in, height=0.17in]{hist_b2563ab10aec.pdf} & Laplace & 89.8 & 85.7 & 80.4 & 79.7\\
\includegraphics[width=0.67in, height=0.17in]{hist_b25646251c8e.pdf} & Normal & 88.5 & 83.3 & 79.7 & 79.4\\
\includegraphics[width=0.67in, height=0.17in]{hist_b25639561798.pdf} & T & 91.3 & 86.6 & 81.4 & 80.2\\
\includegraphics[width=0.67in, height=0.17in]{hist_b2563bff5f61.pdf} & Uniform & 89.5 & 79.8 & 79.9 & 79.6\\
\includegraphics[width=0.67in, height=0.17in]{hist_b25629ff1ad1.pdf} & Beta & 88.3 & 81.5 & 79.6 & 80.4\\
\includegraphics[width=0.67in, height=0.17in]{hist_b2565577c5f2.pdf} & Sparse 1 & 94.2 & 91.3 & 89.6 & 88.4\\
\includegraphics[width=0.67in, height=0.17in]{hist_b2563c24312a.pdf} & Sparse 2 & 90.4 & 87.2 & 83.2 & 82.0\\
\includegraphics[width=0.67in, height=0.17in]{hist_b256123afae3.pdf} & Sparse 3 & 89.3 & 85.3 & 80.8 & 79.7\\
\bottomrule
\end{tabular}

  \caption{\label{Tab:dist_beta} Caption goes here.}
\end{table}

\subsubsection{Epsilon Conundrum}\label{Sec:Epsilon}

Its natural to now consider the performance of Hybrid in the scenario that was the main motivator of the method: the Epsilon conundrum. The set up for the Epsilon Conundrum is essentially just another distribution of $\bb$ values, although one that can prove difficult for methods which fail to take proper precautions. The Hybrid avoids the downfall we observed when applying the Traditional bootstrap to the same simulation set up. Hybrid does, however, over cover due to the fact that the large majority of $\beta$s are near zero (noting the tendency of Hybrid to over cover near zero). But, again, this is to be expected and the important take away is that Hybrid provides viable intervals for values of $\beta$ near zero.

\subsubsection{Selection of \texorpdfstring{$\lambda$}{lambda} (and Estimation of \texorpdfstring{$\sigma^2$}{sigma squared})}

\begin{figure}[hbtp]
  \includegraphics[width=\linewidth]{beta_lambda_heatmap_laplace}
  \caption{\label{Fig:beta_lambda_heatmap_laplace} n = p = 100, where $\beta \sim Laplace(rate = 2)$ and $\X$ generated under independence structure. The red line represents the average CV value of $\lam$. The x-axis was truncated over the range of $\lam_{\CV}$ and presented relative to each simulations $\lam_{\max}$.}
\end{figure}

The reader may be wondering about what happens if the value of $\lambda$ is changed. The data in Figure~\ref{Fig:beta_lambda_heatmap_laplace} comes from a simulation where $\lambda$ was evenly distributed on the $\log_{10}$ scale from $\lam_{\max}$ to $\lam_{\min} = \lam_{\max} * 0.001$. At each value, confidence intervals were obtained and coverage was recorded. This was repeated 100 times, then a gam was fit to provide a smooth estimate of the coverage rate. The relative coverage is defined as estimated coverage minus nominal coverage. The range of $\lambda$ values displayed represents the range of $\lambda_{\CV}$ across the 100 simulations. The red line indicated the average $\lambda_{\CV}$ while the blue line represents the value of $\lambda$ which provided coverage closest to that of nominal.

Depending on the value of $\lam$, values closer to zero see moderate amounts of over coverage whereas there is increasingly greater under coverage as $|\beta|$ increases. The value where this transition occurs (and at which coverage of a specific value of $\beta$ = nominal) varies considerably over the range of $\lam$. At $\lam_{\max}$, this transition occurs at a relatively small $|\beta|$. As the value of $\lam$ decreases from $\lam_{\max}$ this transition occurs at increasingly large values of $\beta$ until all values of $\beta$ have an estimated coverage rate at or above that of nominal. To obtain a coverage rate near that of nominal, $\lam$ needs to be selected such that over and under coverage is balanced. The blue line serves as a general representation of where this tradeoff is met. In this scenario, and in general, $\lam_{\CV}$ does a reasonable job at providing such a balance. However, it should be no surprise that the average $\lam_{\CV}$ (the red line) is to the right of the blue line, as we have seen this pattern previously that the $\lambda$ which minimizes CVE tend to produce over coverage when n = p. That said, the lines on this plot serve more as general representations. Since $\lam_{\max}$ changes which each simulated dataset, the averages are on the values relative to $\lam_{\max}$. Regardless, the balance, of course, depends on the true distribution of $\beta$ (recall that here it is laplace). However, as we will saw in Section~\ref{Sec:Distribution}, $\lam_{\CV}$ also generally does well regardless of the distribution of $\beta$.

It is also important to note that this performance is in spite of needing to provide estimates for the true values of $\lam$ and $\sigma^2$. Figure~\ref{Fig:true_lambda} shows the results of the first simulation described with $\lambda$ set to the true value (left) and with both $\lambda$ and $\sigma^2$ set to their true values (right), see the right plot in Figure~\ref{Fig:nominal_coverage} as a comparison for $\lambda_{\CV}$ and $\sigma^2(\lambda_{\CV})$. When just the true value of $\lam$ is known, there is a slight bit more over coverage (generally $\lam_{\CV}$ is larger than the true value of $\lam$), but otherwise the results are generally comparable to when $\lambda$ is selected via CV. This suggests that $\lam_{\CV}$ generally provides a reasonable selection of $\lam$. However, the differences are more notable when $\sigma^2$ is also set to its true value (1). All the coverages are very near nominal, with slight under coverage for large values of n. This confirms that using CVE to estimate $\sigma^2$, which over estimates of the variability, particularly when n < p, is responsible for the over coverage observed. However, in absence of knowing the true value of $\sigma^2$, over estimating it and maintain coverage above that of nominal is prefered to the alternative.

\logan{This was the last plot I had yet to update with the recent changes (SNR I think is the main driver in the difference seen from what this previously looked like). We get a different story than before.}

\begin{figure}[hbtp]
  \includegraphics[width=\linewidth]{true_lambda}
  \caption{\label{Fig:true_lambda} Caption goes here}
\end{figure}

\subsubsection{Nominal Coverage}

\begin{figure}[hbtp]
  \includegraphics[width=\linewidth]{nominal_coverage}
  \caption{\label{Fig:nominal_coverage} Caption goes here}
\end{figure} 

Figure~\ref{Fig:nominal_coverage} is similar to Figure~\ref{Fig:laplace}, but it focuses only on Hybrid and gives simulation results for three values of n across three different nominal coverage rates. Since the method has a varying coverage rates based on the magnitude of $\beta$, it is important to consider different nominal coverages. Otherwise, it is conceivable that a method could perform well at one coverage rate but not another. However, that is not the case here. Regardless of the nominal coverage, the general pattern remains the same: the method over covers for smaller values of n but coverage converges to the nominal coverage rate relatively quickly. The only difference seen is the compression of this pattern for higher nominal coverage rates.

\begin{figure}[hbtp]
  \includegraphics[width=\linewidth]{zerosample2}
  \caption{\label{Fig:zerosample2} Caption goes here}
\end{figure}

\subsection{Comparison to Ridge Regression CIs}

We start of with a simple example and compare the intervals produced by Zero Sample with that of Ridge. In this example again n = p = 100, however, only one $\beta$ is non-zero. That is, $\beta_{A1} = 1$ and $\beta_{B1}, \beta_{N_1}, \ldots, \beta_{N98} = 0$. Additionally, the data is simulated such that $\rho(\beta_{A1}, \beta_{B1}) = .99$ but all of the N (noise) $\beta$s are uncorrelated with $\beta_{A1}, \beta_{B1}$, and each other.

100 datasets were generated in this manner and each method was applied and respective confidence intervals obtained.

\textbf{Figure A} depicts the results from the simulation with the plot on the right giving box plots for the lower (red) and upper (blue) bounds across the 100 datasets for 3 variables, A1, B1, and N1. On the right, Confidence Intervals for a randomly selected example from the 100 simulated datasets is displayed.

Focusing on A1 and B1, what we would hope to see is wide intervals. Although A1 truly is the signal variable, its high correlation with B1 should produce a large amount of uncertainty about which variable (if not both) contain signal. Ridge, however, fails in this respect in that the intervals are very narrow. On the other hand, Zero Sample displays the desired behavior. The uncertainty entangled in A1 and B1 is reflected in wider intervals. Additionally, intervals for A1 usually do not contain 0 whereas the intervals for B1 contains 0 over 40\% of the time. In general, there is a clear shift in the intervals for A1 compared to B1 suggesting that even with very high correlation, Zero Sample often attributed more of the signal to A1, which was clearly not the case with Ridge.

\begin{figure}[hbtp]
  \includegraphics[width=\linewidth]{highcorr}
  \caption{\label{Fig:highcorr} Caption goes here}
\end{figure}


\subsection{Comparison to Other HDCI Methods}\label{Sec:Comparison}

There are few other methods for obtaining intervals for the lasso and even fewer with implementation in companion R packages to allow for easy usage and comparison. Two that we were able to identify were Selective Inference (SelInf; from \texttt{selectiveInference}) and Bootstrap Lasso Projection (BLP; from \texttt{hdi}).

Neither of these method fits the criteria, remaining faithful to the model fit, that was in mind when working on a new confidence interval producing method. Specifically, the Bootstrap Lasso Projection is also known as the de-sparsified Lasso and by default the method reselects $\lambda$ for each bootstrap sample, making it ambiguous on how the results related to a specific model fit. Selective Inference doesn't directly correct for the bias introduced by penalization but this method does only provide intervals for variables with non-zero coefficients. Additionally, as we will see in Section~\ref{Sec:RDA}, the intervals produced are often questionable in relation to their corresponding point estimates, especially when p is greater than n.

\subsubsection{Similation Study}

\begin{figure}[hbtp]
  \includegraphics[width=\linewidth]{laplace_comparison}
  \caption{\label{Fig:laplace_comparison} Caption goes here}
\end{figure}

\logan{Emphasize that bias comes at the cost of wider intervals and more variable coverage.}

This simulation study compares the performance of the methods with their default parameters. With that said, as implemented in \texttt{hdi}, there is no way to specify the $\lambda$ used for BLP, which defaults to the 1-SE solution from \texttt{cv.glmnet} (and, again, which is re-selected for each bootstrap draw). For the simulation study, we use this ``out of the box'' setup, however, it was unsatisfactory for the comparison in the subsequent data analyses. For the data analyses, we forked the \texttt{hdi} repository and adjusted it to allow for the specification of $\lambda$. Specifically, to allow for the use of the $\lambda$ which minimizes CVE. We did attempt to use this version in the simulation study, but this highlighted a limitation for BLP which made such usage infeasible under the desired setup. Specifically, BLP only works when the number of non-zero coefficients is less than n, something that is more likely to occur for the 1-SE solution. As the simulation is set up, this proved prohibitive as more often than not when n was small relative to p, the method resulted in an error. Although it would have been nice to have the option to compare the methods under the same value of $\lambda$, there is something to be said for comparing the performance of the methods as closely as possible to what a user would experience if they started using one of the methods with its default arguments. However, it must be emphasized that this means the $\lambda$ values used for BLP vary greatly from those used for SelInf and Hybrid\footnote{\texttt{cv.glmnet} and \texttt{cv.ncvreg} generally select a similar value for $\lambda$. \texttt{cv.glmnet} was used for SelInf since it was what was used in examples for the \texttt{selectiveInference} package and \texttt{cv.ncvreg} was used for Hybrid, since the methods developed in this manuscript are implimented in \texttt{ncvreg}.}, which both use a single $\lambda$ for each dataset with the $\lambda$ being the value which minimizes CVE from \texttt{cv.glmnet} and \texttt{cv.ncvreg}, respectively, applied to the original dataset. HDI, on the other hand, has B $\lambda$ values each chosen using the 1-SE $\lambda$ from \texttt{cv.glmnet} applied to each of the B bootstrap draws for each dataset and set of intervals produced.

The simulation results presented here are from a set up identical to that described in Section~\ref{Sec:Coverage}. In fact, the results for Hybrid are the same as used for the earlier figures, just displayed differently. 

Referencing Figure~\ref{Fig:laplace_comparison}, both BLP and SI initially appear to perform strikingly well. Both have coverage very near that of nominal and lack the pattern seen with the Hybrid method and instead provide rather consistent coverage regardless of the magnitude of $\beta$. However, Figure~\ref{Fig:laplace_other} tells a different story. First, although BLP provides average coverage rates the closest to $80\%$, there were a number of cases where the coverage dipped significantly. Potentially more concerning, with $n = 400$, the coverage noticeably drops below 80\%. Although the lasso is generally thought of a model for high dimensional data, it is used across the entire spectrum of datasets, so it would be preferred to see convergence towards the nominal rate of coverage.

\begin{figure}[hbtp]
  \includegraphics[width=\linewidth]{laplace_other}
  \caption{\label{Fig:laplace_other} Caption goes here}
\end{figure}

The behavior for the Hybrid Bootstrap has been covered previously, specifically that it generally over covers when $n < p$ but has coverage very near nominal as n increases above p. Additionally, although not immune to under coverage, it performed more reliably than the other two methods.

\begin{table}[hb]
  \centering
  \begin{tabular}{cccc}
  \hline
  & \multicolumn{2}{c}{Simulations} & Variables \\
  n & \# Succeeded & \# Non-finite Median Width & \# Included on Average \\
  \hline
  50  & 80 & 18 & 18.5 \\
  100 & 94 & 12 & 30.4 \\
  400 & 100 & 3 & 70.1 \\
  \hline
  \end{tabular}
  \caption{Selective Inference Results. A majority of the errors occur due to too large of a $\lambda$ value being selected using CV (causing all coefficients to equal zero), however, a failure to satisfy the polyhedral constraint also was a source of errors.}
  \label{Tab:selective_inference}
\end{table}

Before moving onto the results for SelInf, there are a couple numbers not in Figure~\ref{Fig:laplace_comparison} which are also important to consider and which are provided in Table~\ref{Tab:selective_inference}. When n = 50, 20 of the 100 simulations errored out, with only 6 when n = 100, and none when n = 400. The last column gives the average number of variables that were selected  (had confidence intervals) of the simulation iterations that did not produce errors. It is this subset that is included in the coverage plot for SI. The fact that SelInf does not produce intervals for all variables likely explains the sporadic coverage behaviors, since its coverage values are averaged over fewer variables. There are also a considerable number of iterations that had coverage below 50\% coverage, an issue which does not seem to be remedied even with larger values of n. So, even if on average, SelInf looks good, the underlying behavior is much less desirable.

Staying with SI but directing our attention towards the interval widths, the concern surrounding this method only grows. Of the 80 that simulations that suceeded for n = 50, the median width of the intervals produced (from the variables included in the model, i.e. column 3) was infinite for 18 of the simulated datasets. For n = 100, 12 of the 94 had infinite median widths. By n = 400, only 3 of the simulations had infinite median widths. However, even when the medians were finite, they were nearly always extremely wide, even for n = 400. This behavior was also observed when applying the method to real data sets which will be covered in Section~\ref{Sec:RDA}. BLP and Hybrid on the other hand produce intervals which are more similar in width although BLP does tend to produce wider intervals and with a bit more variability in the width.

The runtime of the three methods also differs considerably with SI being the fastest and BLP being by far the slowest with about an order of magnitude difference separating the methods respectively. Additionally with BLP, there is an odd non-monotonic behavior which we looked into briefly but could not find an explanation for. This behavior occurred in all reruns of the same simulation. Hybrid and SI both have a monotonically increasing relationship with sample size, although Hybrid is more affected by the increasing sample size. In our testing, speed was not a concern for SI, was noticeable for Hybrid, and prohibitive for BLP which will be returned to in Section~\ref{Sec:Scheetz2006}. BLP is particularly slow with its default arguments because of the reselection of $\lambda$ (using CV) for each iteration of the bootstrap.

\logan{I could see the reader questioning... well if you overcame a similar issue (just noted how many errors occured) for SelInf... why not do the same for BLP? The issue for BLP however is far worse because an error can potentially arrise in any bootstrap iteration whereas SelInf has to work just once, for the original dataset. That is, BLP fails much more miserably.}

\logan{I did start to track down the error for BLP and it seems to be due to a NaN showing up where they weren't expecting and it not being handled properly. However, I think it would take a considerable amount of time to find the root of the error and potentially fix it... which I could do but not sure if it is worth the effort.}

\subsection{Real Data Analysis}\label{Sec:RDA}

We conclude the results section by considering the intervals produced by the Hybrid bootstrap applied to two datasets: \texttt{whoari} (World Health Organization study on acute respiratory illnesses) and \texttt{Scheetz2006} (Gene expression in the mammalian eye). These two datasets represent two extremes in terms of dimensionality. \texttt{whoari} contains 816 observations and 66 features. However, at the other end, \texttt{Scheetz2006} is certainly a high dimensional dataset, containing just 120 observations but with 18975 features. Additionally we compare the intervals produced by the Hybrid bootstrap to those of BLP and SI.

As mentioned in the previous section, we wanted to the results for the real data analysis to correspond directly back to a single set of point estimates for all three methods. The $\lambda$ of interest is the value which minimizes CVE. For the purposes of the real data analyses, this $\lambda$ value was selected using \texttt{cv.glmnet}. Since \texttt{hdi} was not set up with the flexibility to allow for the specification of $\lambda$, we forked the \text{hdi} repo and made the necessary modifications. So, if one would want to obtain the same results, they would also need to take the same steps. It should not be ignored that BLP does provide its own estimates. However, the connection to the the lasso is obscured, so for the purpose of this comparison we use the the same point estimates for the three methods.

For the comparison of these three methods, we want to put a clear emphasis on how the intervals relate to the corresponding estimates from the lasso fit using a selected value of $\lambda$. This is not an issue for Hybrid or SI, which both allow $\lambda$ to be specified, but does require another adjustment for BLP, but one that is supported in the arguments. This adjustment is to set \texttt{boot.shortcut = TRUE}. From hdi's documentation, if \texttt{boot.shortcut = TRUE}, ``the lasso is not re-tuned for each bootstrap iteration, but it uses the tuning parameter computed on the original data instead.'' From their documentation it is not clear what value of $\lambda$ is used, however, as previously noted, by searching through their code, one can determine it is the $\lambda$ corresponding to the 1-SE solution from \texttt{cv.glmnet}.

\subsubsection{World Health Organization study on acute respiratory illnesses (whoari)}

The \texttt{whoari} dataset comes from the study ``Development of a clinical prediction model for an ordinal outcome: the World Health Organization multicentre study of clinical signs and etiological agents of pneumonia, sepis and meningitis in young infants'' written by \cite{Harrell1998}. The study considered a few acute illness in young infants across several countries, and the dataset used here is a subset of 816 infants who presented with pneumonia in the country Ethiopia. As alluded to in the title, collection of this data was done with the intention of building a prediction model to assess the severity of an infant presenting with a serious infection, which represents the main cause of mobility and mortality in infants under 3 months in these developing countries. Diagnosis of severity is a difficult task, and developing rules for grading the severity of disease is important for prompt delivery of treatment for those who need it while also avoiding unnecessarily and costly treatments where possible. The outcome considered here is ordinal (taking on a number from 1 - 5), however, for simplicity we treat the outcome as continuous and feel that the results are reasonable, at least for a comparison of the three methods under consideration. The variables collected contain information on vital signs, family history, and clinical observations and represent a range of datatypes from binary to ordinal to continuous. With $N \approx 10p$, this dataset is not necessarily high dimensional, but sits on the edge of where classical methods and their resulting inferences may be questionable. Additionally, the use of a model that produces sparsity is beneficial both for interpretation and for ultimately determining factors for assessment in practice, where obtaining predictions from a model may be prohibitive.

Figure~\ref{Fig:comparison_data_whoari} provides the confidence intervals and corresponding point estimates. The point estimates are the same across all three methods and come from the lasso fit on the original data with the $\lambda$ selected using \texttt{cv.glmnet}. It is important to emphasize that the range of the x-axis is different for each of the plots corresponding to the three methods. The Hybrid produces the narrowest intervals and SelInf produces by far the widest. Despite the difference in widths, Hybrid and BLP share similar patterns, however, the conclusions that might be drawn could conceivably be quite varied. This may most easily be observed by considering a common point of interest: whether or not an interval includes zero. Three intervals from BLP do not contain zero, 13 intervals from SelInf do not contain zero, and 21 from Hybrid do not contain zero. That SelInf produces 13 intervals not containing zero is only part of the picture, since it is also important to note SelInf selects (and provides intervals for) 37 of the 66 variables and only produces one interval here with an infinite bound (for \texttt{abb}). That said, there are a number of intervals that are unreasonably wide and we believe it would be difficult to provide a convincing interpretation for the intervals produced by SelInf.

The comparison between Hybrid and HDI is more interesting since, visually, they appear very similar. The patterns observed here are similar to those in the preceding simulation study. As previously mentioned, the intervals from Hybrid are narrower on average. Additionally, while Hybrid's intervals, relative to the point estimates tend to be skewed towards zero or symmetric, BLP's intervals often are skewed away from zero, likely as a result of debasing. It appears then, that the higher coverage at larger values of $\beta$ as seen in the simulation is likely due to a combination of increased width and the skewness induced by debasing.  

\begin{figure}[hbtp]
  \includegraphics[width=\linewidth]{comparison_data}
  \caption{\label{Fig:comparison_data_whoari} Caption goes here}
\end{figure}

\subsubsection{Gene expression in the mammalian eye (Scheetz2006)}\label{Sec:Scheetz2006}

The \texttt{Scheetz2006} data was obtained from the study ``Regulation of gene expression in the mammalian eye and its relevance to eye disease'', written by \cite{Scheetz2006}. This study involved measuring the RNA levels from the eyes of 120 rats. Of 31000 different probes used, 18976 were detected at a sufficient level to be considered ``expressed''. For this analysis we treat one of the genes, Trim32, as the outcome since it is known to be linked to the genetic disorder, Bardet-Biedl Syndrome (BBS). The remaining 18975 genes are used as covariates with the goal of determining other genes which may have expression correlated with Trim32 and thus also may contribute to BBS. 

In the simulation study, we saw that when p is large relative to n, SelInf's difficulties are amplified. However, applied to \texttt{Sheetz2006}, where $p > 100n$, the issues are unignorable. SelInf provides intervals for 66 of the 18975 features however, every single one of them has a lower or upper bound that is infinite. Additionally, for the bounds that are finite, they are all also extremely large compared to their respective point estimates and in comparison to the intervals produced by Hybrid and BLP. Additionally, none of the intervals contains zero. Potentially even more odd is that of the 66 intervals, 62 of them were completely of the opposite sign as the corresponding estimate. 

Like with \texttt{whoari}, Hybrid and BLP produce similar intervals with the characteristic differences more prominent due to the dimensionality of the problem. The intervals of Hybrid are narrower. Additionally, from the examples provided in Figure~\ref{Fig:comparison_data_scheetz}, the intervals produced by BLP may appear more desirable. With this very high dimensional dataset, Hybrid produces intervals that are noticeably drawn in towards zero, likely due to the larger penalty selected in this setting. As such, a couple of the intervals produced by Hybrid exclude their corresponding point estimate. 

Despite these differences, depending on the perspective taken, there is not a large discrepancy for the variables deemed significant by the two methods. BLP produces 6 intervals which exclude zero, while Hybrid produces 3. This is twice as many, but in the grand scheme of nearly twenty-thousand variables, this is a relatively minor difference. Both methods have intervals not containing zero for \texttt{1389910\_at}, \texttt{1378319\_at}, and \texttt{1385395\_at}, while BLP produces intervals three additional genes which do not contain zero.

That said, we emphasize again that a user is not able to obtain these results from the implementation in \texttt{hdi}. Additionally, even with \texttt{boot.shortcut = TRUE}, on a MacBook Pro with 16 GB of RAM and an Apple M1 Pro chip, BLP took over 6 hours to run. \logan{Should update with run on HPC?}

\begin{figure}[hbtp]
  \includegraphics[width=\linewidth]{comparison_data_scheetz}
  \caption{\label{Fig:comparison_data_scheetz} Caption goes here}
\end{figure}

\section{Discussion}

Estimation of $\sigma^2$.

\subsection{Space Requirements}

As implemented, a clear limitation of this method is that it takes a numeric matrix size $B \times p$. With $B = 1000$, the sample matrix gets large enough to cause memory concerns even when $p$ is on the order of $1e5$. For many datasets, this is likely not of concern. However, given that lasso is often used for datasets where $p$ is large, it is clearly not an edge case where $p$ is of this or larger order. One could reduce the size of the sample matrix by reducing the number of draws, but this is unsatisfactory and produces little additional leeway for what seems like a unacceptable sacrifice. This is an ongoing are of interest and a valuable areas of research for any high dimensional bootstrap methods. One solution would be using incremental quantile estimation such as the method introduced by \cite{Tierney1983}. An alternative option would be deriving a method with similar properties but which relies on samples statistics that are relatively straightforward to update over sequential samples (i.e. mean and variance).

\subsection{Remember the Name}

Since the main method suggested here falls somewhere between the Traditional Bootstrap and the Posterior Bootstrap, we propose the full name as the Posterior Adjusted Traditional Hybrid (PATH) Bootstrap.

\newpage

\section*{Supplement}

\subsection{Supplement A: Sampling from the Full Conditional Posterior}\label{Sup:A}

\logan{Discuss notation}

Recall, the lasso can be formulated as a Bayesian regression model with the prior $\p(\bb) = \prod_{j = 1}^{p} \frac{\gamma}{2}\exp(-\gamma \abs{\beta_j})$ for $\gamma > 0$ and with $\lam = \gamma \frac{\sigma^2}{n}$. We also saw the likelihood (written here in terms of the partial residuals) is proportional to $\exp(-\frac{n}{2\sigma^2} (\beta_j^2 - 2z_j\beta_j))$ where $z_{j} = \frac{1}{n} \x_{j}^{T}\r_{-j}$.  With this the form of the full conditional posterior can be worked out as follows:
\as{
\Rightarrow P(\beta_j | \r_{-j}) &\propto \exp(-\frac{n}{2\sigma^2} (\beta_j^2 - 2z_{j}\beta_j)) \frac{n \lambda}{2 \sigma^2} \exp(-\frac{n \lambda} {\sigma^2} \abs{\beta_j}) \\
&\propto \exp(-\frac{n}{2\sigma^2} (\beta_j^2 - 2 z_j\beta_j +  2 \lambda \abs{\beta_j})) \\
&= \exp(-\frac{n}{2\sigma^2} (\beta_j^2 - 2(z_j\beta_j - \lambda \abs{\beta_j}))) \\
&=
\begin{cases}
\exp(-\frac{n}{2\sigma^2} (\beta_j^2 - 2(z_j + \lambda)\beta_j)), \text{ if } \beta_j < 0, \\
\exp(-\frac{n}{2\sigma^2} (\beta_j^2 - 2(z_j - \lambda)\beta_j )), \text{ if } \beta_j \geq 0 \\
\end{cases} \\
&\propto
\begin{cases}
C_{-} \exp(-\frac{n}{2\sigma^2} (\beta_j - (z_j + \lambda))^2), \text{ if } \beta_j < 0, \\
C_{+} \exp(-\frac{n}{2\sigma^2} (\beta_j - (z_j - \lambda))^2), \text{ if } \beta_j \geq 0 \\
\end{cases}
}
where, $C_{-} = \exp(\frac{z_j \lambda n}{\sigma^2})$ and $C_{+} = \exp(-\frac{z_j \lambda n}{\sigma^2})$.

At this point, te reader likely notices that the piecewise defined full conditional posterior is made up of a kernel of two normal distributions. This can be leveraged and draws can be efficiently obtained from through a mapping onto the respective normal distributions. To define this mapping, it helps to introduce a concept and some notation. First, the use of ``tails'' in this supplement refers to the entirety the a distribution between zero and $\pm \infty$. That is, the lower tail is any part of the distribution below zero and the upper tail is any part greater than zero and $P(X \in lower \cup X \in upper) = 1$. Accordingly, we will let the tail probabilities in each of the two normals to transformed on to be denoted $Pr_{-}$ and $Pr_{+}$ respectively and the probability in each of the tails of the posterior, denoted $Post_{-}$ and $Post_{+}$ respectively. $Pr_{\pm}$ is trivial to compute with any statistical software. $Post_{\pm}$ is conceptually simple, although care must be taken to avoid numerical instability as n increases. Now, as noted,
\as{
P(\beta_j | \r_{-j})  & \propto
\begin{cases}
C_{-} Pr_{-}, \text{ if } \beta_j < 0, \\
C_{+} Pr_{+}, \text{ if } \beta_j \geq 0\\
\end{cases}
} which implies that $Post_- = \frac{C_{-} Pr_{-}}{C_{-} Pr_{-} + C_{+} Pr_{+}}$ and similarly for $Post_+$. However, to avoid numerical instability, or at least handle it properly when it is unavoidable, we need to work on the $\log$ scale. This works well for most of the problem, but computation of $Post_-$ and $Post_+$ need something a bit more since, for example, $\log(Post_-) = \log(C_{-}Pr_{-}) - \log(C_{-} Pr_{-} + C_{+} Pr_{+})$. That is, the denominator still must be computed then the $\log$ taken which does not allow operation on the $\log$ scale to fully address potential instability. Instead, $\log(Post_-)$ can be computed with $\log(C_-Pr_-) -  \log(C_+Pr_+) - \log(1 + \exp(\log(C_-Pr_-) -  \log(C_+Pr_+)))$. This still doesn't completely address the issue, however, if $\exp(\log(C_-Pr_-) -  \log(C_+Pr_+))$ is infinite then $C_-Pr_- >> C_+Pr_+$ and $\log(Post_-) \approx 0$.

With these values, we can compute quantiles by mapping the corresponding probabilities $p$ for the posterior onto the probabilities $p^*$ for the corresponding normals. Which normal the quantiles of interest ultimately come from is determined based on $Post_{\pm}$. For example, if $Post_{+} = 0.98$ and $p = 0.1$ the $p$ would be mapped onto the positive normal. As one more example, say $Post_{+} = 0.4$ and $p = 0.5$, then $p$ would be mapped onto the negative normal. The transformation to map a given probability from the posterior depends on which tail the quantile resides in on the posterior (equivalently which normal it is being mapped to, the positive or negative). This map is simply:

\as{
p^* &= p \times (Pr_{\pm} / Post_{\pm}) \\
}


Once the respective probabilities are mapped, one can simply use the inverses of the normal CDFs that the probabilities were mapped to. That being said, there is a nuance worth pointing out. When transforming the probabilities, the step to determine which tail the respective quantile comes from occurs first. With this, the probability should be adjusted so that it refers to the probability between the quantile of interest and the respective tail. After this, then the transformation can be applied. With that, obtaining draws from the full conditional posterior can be summarized as follows (written for a single $\beta$ for simplicity):

\begin{enumerate}
  \item Select $\lambda$, fit lasso and obtain estimates corresponding to $\lambda$, estimate $\sigma^2$
	\item Obtain the partial residuals, $\r_{-j}$, and compute $z_j$
	\item Compute $Pr_{-}$ = $\Phi(0, z_j + \lam, \frac{\sh^2}{n})$ and $Pr_{+}$ = $1 - \Phi(0, z_j - \lam, \frac{\sh^2}{n})$
	\item Compute $Post_-$ and $Post_+$ as detailed above
	\item Obtain the quantile $(q)$ corresponding to the given probability $(p)$ of interest:
  \begin{algorithmic}
    \If {$p \leq Post_{-}$}
      \State $q = \Phi^{-1}(p(Pr_{-} / Post_{-}), z_j + \lam, \frac{\sh^2}{n})$
    \Else
        \State $q = \Phi^{-1}(1 - (1 - p)(Pr_{+} / Post_{+}), z_j - \lam, \frac{\sh^2}{n})$
    \EndIf
  \end{algorithmic}
\end{enumerate}
  
\newpage