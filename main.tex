\section{Introduction}

The prevailing opinion is that the traditional use of bootstrapping to produce confidence intervals for penalized regression is suboptimal due to the inherent behavior of estimates produced by penalization.

Specifically, we see two potential problems that arise when proposing to use bootstrapping for penalized regression, we will refer to these obstacles as,

\begin{enumerate}
\item the Bias Tradeoff, and
\item the Epsilon Conundrum.
\end{enumerate}

The Bias Tradeoff refers to the inherent bias introduced by penalization in order to be able to fit oversaturated models. This penalization leads to coverages much higher than nominal coverage rates for coefficients with true values at or very near zero and leads to much lower coverage when coefficients are non-zero. We argue that coverage in the traditional sense is too ridged of a paradigm to apply to penalized regression inference. Instead, we offer a different perspective and argue that the impact that penalization induced bias has is an inherent feature of penalized regression rather than a flaw. Since high dimensional problems often necessitate such an alternate approach, we offer guidance on how to interpret the confidence intervals for lasso as presented in this paper. Additionally, we compare the proposed method to two other inferential methods which take a different perspective.

The second problem, the Epsilon Conundrum, is also related to the shrinkage but arises as an arguably more disturbing manifestation: confidence intervals of length zero or, more often, with a single endpoint that is exactly zero. Some penalized regression methods, particularly the lasso, often result in a sparse solution. If using a traditional quantile based bootstrap confidence interval, this will lead to an interval of [0,0] if a given variable is rarely or never included in the active set. As the dimensionality of the problem grows, this becomes an increasing occurrence leading to a large majority of intervals possessing a length of zero. When the true value of the coefficient is 0, the interval at least contains the truth. However, this issue is particularly troublesome when one considers what happens when the true value is not precisely zero, as is likely the case in most reasonable scenarios. By just shifting the true value by $\eps$, an immediate drop in coverage would occur, hence the name: the Epsilon Conundrum.

The later of these two problems is addressed in our novel approach to producing bootstrap based confidence intervals for the lasso.

\section{Lasso Bootstrap Confidence Intervals}

In what follows, we examine four different methods for obtaining confidence intervals via bootstrapping.

The first is the traditional bootstrap which results simply from taking the point estimate for each $\beta_j$ for each bootstrap sample. As mentioned previously and as we will see, this is has clear pitfalls which the subsequent three alternatives attempt to address.

The second is the sample bootstrap. This leverages the Bayesian formulation of the lasso. For each bootstrap sample, a random draw from the $O_1$ posterior os obtained instead of taking the point estimate (equivalent to the mode) of the posterior. \logan{discuss}

The sample bootstrap tends to be too aggressive in the additional variability introduced. As such, the third (called zero sample) falls in between the first and the second and only samples from the $O_1$ posterior if the point estimate for a given $\beta_j$ is equal to zero for that bootstrap sample.

The fourth and final method is a niave approach at producing debiased intervals and as such is called debiased. Instead of taking the point estimate as the traditional method would, this method uses the correlation between the partial residuals obtained from the lasso with $\boldsymbol{x}_j$. As such, one might expect this to address both issues mentioned above. As we will see, it does in fact, however, its performance in general is subpar when $p < n$.

\subsection{An Overview of the Four Alternatives}

As with other penalized regression models, the lasso can be formulated as a Bayesian regression model by setting an appropriate prior. For the lasso, this prior is a Laplace, also called double-exponential, distribution:

\as{\p(\bb) = \prod_{j = 1}^{p} \frac{\gamma}{2}\exp(-\gamma \abs{\beta_j}), \gamma > 0}

With this prior, the lasso estimate $\bbh(\lam)$ is the posterior mode of $\bb$ when $\lam = \gamma \frac{\sigma^2}{n}$.

We propose leveraging this relationship to build confidence intervals for $\bb$. Specifically, we propose the following process for given values of $\lam$, $\sigma^2$, and significance level, $\l$:

\begin{enumerate}
\item Obtain a bootstrap sample of $\X$ and $\y$
\item For each $\beta_j$:
	\begin{enumerate}
	\item Obtain the first-order, $O_1$, posterior for $\beta_j$
	\item Draw r randomly selected quantiles and save
	\end{enumerate}
\item Repeat m times
\item For each $\beta_j$, combine the r*m samples and compute the quantiles for $p_L = (.5 - \l/2)$ and $p_U = (.5 + \l/2)$
\end{enumerate}

\subsection{Sampling from the Full Conditional}

In this section, we describe the details and important considerations for the sampling performed in methods two and three above.

Unfortunately, a Normal likelihood and Laplace prior are not conjugate and in general, the absolute value in the exponent of the Laplace makes many common manipulations for the posterior more difficult. Luckily, however, the marginal posterior can be shown to be a mixture of a right and left truncated normal where the truncation occurs at zero for right and left tails respectively. To obtain a $O_1$ posterior, we use the partial residuals, $\r_{-j}$, in the likelihood. This is natural due to the common CD algorithms used to arrive at lasso estimates. In the steps that follow, we will assume that $\X$ has been standardized s.t. $\x_j^T\x_j = n$. For $\beta_j$,

\as{
L(\beta_j | \r_{-j}) &= (\sigma \sqrt{2\pi})^{-n} \exp(-\frac{1}{2\sigma^2} (n\beta_j^2 - 2\x_{j}^{T}\r_{-j}\beta_j + \r_{-j}^{T}\r_{-j})) \\
&\propto \exp(-\frac{n}{2\sigma^2} (\beta_j^2 - 2z_j\beta_j)), \text{ where } z_{j} = \frac{1}{n} \x_{j}^{T}\r_{-j} \\
\Rightarrow P(\beta_j | \r_{-j}) &\propto \exp(-\frac{n}{2\sigma^2} (\beta_j^2 - 2z_{j}\beta_j)) \frac{n \lambda}{2 \sigma^2} \exp(-\frac{n \lambda} {\sigma^2} \abs{\beta_j}) \\
&\propto \exp(-\frac{n}{2\sigma^2} (\beta_j^2 - 2 z_j\beta_j +  2 \lambda \abs{\beta_j})) \\
&= \exp(-\frac{n}{2\sigma^2} (\beta_j^2 - 2(z_j\beta_j - \lambda \abs{\beta_j}))) \\
&=
\begin{cases}
\exp(-\frac{n}{2\sigma^2} (\beta_j^2 - 2(z_j + \lambda)\beta_j)), \text{ if } \beta_j < 0, \\
\exp(-\frac{n}{2\sigma^2} (\beta_j^2 - 2(z_j - \lambda)\beta_j )), \text{ if } \beta_j \geq 0 \\
\end{cases} \\
&\propto
\begin{cases}
C_{-} \exp(-\frac{n}{2\sigma^2} (\beta_j - (z_j + \lambda))^2), \text{ if } \beta_j < 0, \\
C_{+} \exp(-\frac{n}{2\sigma^2} (\beta_j - (z_j - \lambda))^2), \text{ if } \beta_j \geq 0 \\
\end{cases}
}

Where, $C_{-} = \exp(\frac{z_j \lambda n}{\sigma^2})$ and $C_{+} = \exp(-\frac{z_j \lambda n}{\sigma^2})$.

Ignoring the truncation for a moment and focusing on the kernels individually, we can see that the two distributions above are what we would expect from the lasso solution, having a mean $z_j + \lambda$ and $z_j - \lambda$ respectively and a variance of $\frac{\sigma^2}{n}$. That said, due to the truncation, the variance of the posterior is less than or equal to this (plot the $O_1$ posterior and this should become immediately clear). Furthermore, a mapping allows tail probabilities from the posterior to be translated to probabilities onto corresponding known normal distributions ($\N(0, z_j \pm \lambda, \frac{\sigma^2}{n})$). For this translation, we need two pieces of information:

\begin{enumerate}
\item The non-truncated probability in each of the two normals to transformed on to, denoted $Pr_{-}$ and $Pr_{+}$ respectively.
\item The probability in each of the tails of the posterior, denoted $Post_{-}$ and $Post_{+}$ respectively.
\end{enumerate}

(1) is trivial to compute with any statistical software. Similarly, (2) is conceptually simple, although care must be taken to avoid the pitfall of numerical instability introduced as n increases.

To reduce the number of computational steps, one may note that:

\as{
P(\beta_j | \r_{-j})  & \propto
\begin{cases}
C_{-} Pr_{-}, \text{ if } \beta_j < 0, \\
C_{+} Pr_{+}, \text{ if } \beta_j \geq 0 \\
\end{cases}
}

Which implies that $Post_- = \frac{C_{-} Pr_{-}}{C_{-} Pr_{-} + C_{+} Pr_{+}}$ and similarly for $Post_+$. However, to avoid numerical instability, or at least handle it properly when it is unavoidable we need to work on the $\log$ scale. This works well for most of the problem, but computation of $Post_-$ and $Post_+$ need something a bit more since, for example, $\log(Post_-) = \log(C{-}Pr_{-}) - \log(C_{-} Pr_{-} + C_{+} Pr_{+})$. That is, the denominator still must be computed then the $\log$ taken which does not allow operating on the $\log$ scale to fully address potential instability. Instead, $\log(Post_-)$ can be computed with $\log(C_-Pr_-) -  \log(C_+Pr_+) - \log(1 + \exp(\log(C_-Pr_-) -  \log(C_+Pr_+)))$. This still doesn't completely address the issue, however, if $exp(\log(C_-Pr_-) -  \log(C_+Pr_+))$ is infinite then $C_-Pr_- >> C_+Pr_+$ and $\log(Post_-) \approx 0$.

With these values, we can compute quantiles by mapping the corresponding probabilities $p$ for the posterior onto the probabilities $p^*$ for the corresponding normals. Which normal the quantiles of interest ultimately come from is determined based on the values in (2). For example, if $Post_{+} = 0.98$ and $p = 0.1$ the $p$ would be mapped onto the positive normal. As one more example, say $Post_{+} = 0.4$ and $p = 0.5$, then $p$ would be mapped onto the lower normal. The transformation to map a given probability from the posterior depends on which tail the quantile resides in on the posterior (equivalently which normal it is being mapped to, the positive or negative). This map is simply:

\as{
p^* &= p \times (Pr_{\pm} / Post_{\pm}) \\
}

Once this respective probabilities are mapped, one can simply use the inverse of the normal CDFs that the probabilities were mapped to. That being said, there is a nuance worth pointing out. When transforming the probabilities, the step to determine which tail the respective quantile comes from occurs first. With this, the probability should be adjusted so that it refers to the probability between the quantile of interest and the respective tail. After this, then the transformation can be applied.

This solution corresponds to a particular value of $\lam$ and $\hat{\sigma}^2$. The only thing prohibitive to producing intervals for the entire lasso path is computational resources since the bootstrap is already computationally expensive. That brings our decision down to producing an estimate for $\sigma^2$. Our recommendation is to first use cross validation (CV) once to produce an estimate. However, note that producing an estimate in this manner implicitly depends on $\lam$, so new estimates for $\sigma^2$ should be obtained for each value of $\lam$.

Our recommendation is to use CV to select $\lam$ and produce an estimate for $\sigma^2$ and produce bootstrap confidence intervals corresponding to these values to report.

All together this leads to the following steps to perform the bootstrap (presented on original scale for compactness):

\begin{enumerate}
\item Perform CV using the original data to select $\lam$ and estimate $\hat{\sigma}^2(\lam)$
\item For i $\in \lbrace 1 \ldots m \rbrace$:
\begin{enumerate}
\item Obtain a pairs bootstrap sample, $\X^*$ and $\y^*$
\item Fit lasso with $\X^*$ and $\y^*$
\item For j $\in \lbrace 1, \ldots, p \rbrace$:
	\begin{enumerate}
	\item Obtain the partial residuals, $r_{-j}$ and compute $z_j$
	\item Compute $Pr_{-}$ = $\Phi(0, z_j + \lam, \frac{\sh^2}{n})$ and $Pr_{+}$ = $1 - \Phi(0, z_j - \lam, \frac{\sh^2}{n})$
	\item Compute $Post_-$ and $Post_+$
	\item For $r$ $\in \lbrace 1, \ldots, R \rbrace$:
	\begin{algorithmic}
		\If {$p_r \leq Post_{-}$}
			\State $q_r = \Phi^{-1}(p_r(Pr_{-} / Post_{-}), z_j + \lam, \frac{\sh^2}{n})$
		\Else
			\State $q_r = \Phi^{-1}(1 - (1 - p_r)*(Pr_{+} / Post_{+}), z_j - \lam, \frac{\sh^2}{n})$
		\EndIf
	\end{algorithmic}
	\item Save $\q$
	\end{enumerate}
\end{enumerate}
\item For each $\beta_j$, combine the $r*m$ samples and compute the quantiles for $p_L = (.5 - \l/2)$ and $p_U = (.5 + \l/2)$ to produce a final confidence interval estimate
\end{enumerate}

\section{Interval Behavior}

Need to make note on the rate used for the laplace. Which will come back into play with more or less sparse. Add extra plots to appendix for now.

\subsection{Comparing Lasso Bootstrap Methods}

Need to say something (and show) about the full conditional posterior itself not being enough

\includegraphics[width=\linewidth]{laplace.pdf}

Say something about rate choice being arbitary under fixed SNR

\includegraphics[width=\linewidth]{laplace_width_bias.pdf}

\subsection{Nominal Coverage}

\includegraphics[width=\linewidth]{nominal_coverage.pdf}

\subsection{True lambda / sigma}

\includegraphics[width=\linewidth]{true_lambda.pdf}

\section{Sensitivity}

\subsection{Correlation Structures}

\includegraphics[width=\linewidth]{correlation_structure.pdf}

\subsection{Selection of \texorpdfstring{$\lambda$}{lambda}}

\includegraphics[width=\linewidth]{beta_lambda_heatmap_laplace.pdf}

\subsection{Distribution of Beta}

Sparse: more at zero, little moderate, heavier tails

Sparse 1: 0 x 90, +/- $(0.5 x 3, 1, 2)$,
- high, low, low  - brown
Sparse 2: 0 x 70, 30 $\sim N(0, 1)$
- high, moderate, low - purple
Sparse 3: 0 x 50, 50 $\sim N(0, 1)$
- moderate, moderate, low - yellow
Laplace: 100 $\sim dexp(2)$
- moderate, high, high - green
T: 100 $\sim T(3)$
- low, high, high - blue
Normal: 100 $\sim N(0, 1)$
- low, high, low - red

\includegraphics[width=\linewidth]{distribution_of_beta.pdf}

Here is a terrible plot just to give a general idea of them :

\includegraphics[width=\linewidth]{sparsity.pdf}

\subsection{Epsilon Conundrum (Comparison to Traditional Quantile Bootstrap)}

\includegraphics[width=\linewidth]{epsilon_conundrum.pdf}

\subsection{Correlation}

\subsection{Correlated Features (Comparison to Ridge)}

\includegraphics[width=\linewidth]{highcorr.pdf}

\section{Comparison to Other Methods}

\begin{enumerate}
\item Selective inference
\begin{enumerate}
\item Does not work for $p > n$ case
\item Only produced CI for variables in active set
\item Seems cumbersome and unstable as implemented
\item Unclear recommended selection of $\lam$
\item Non-finite endpoints not uncommon
\end{enumerate}
\item{Bootstrap Lasso Projection}
\end{enumerate}

\subsection{Comparison Similation}

\includegraphics[width=\linewidth]{laplace_comparison.pdf}

\includegraphics[width=\linewidth]{laplace_other.pdf}

\section{Application: Acute Respiratory Illness (WHO)}

\includegraphics[width=\linewidth]{comparison_data.pdf}

\section{Application: Gene expression in the mammalian eye}

\includegraphics[width=\linewidth]{comparison_data_scheetz.pdf}

\section{Discussion}

\subsection{Space Requirements}

As implimented, this method takes a numeric matrix size $Br \times p$. With $B = 100, r = 10$, the sample matrix gets too large enough to cause memory concerns even when $p$ is on the order of $1e5$. For many datasets, this is likely not of concern. however, given that lasso is often used for datasets where $p$ is large, it is clearly not an edge case where $p$ exceeds the order of $1e5$. One could reduce the number size of the same matrix by reducing the number of draws, i.e. let $r = 1$, but this is unstaisfactoy and produces little additional leeway for what seems like a unnacceptable sacrifice. What we propose instead is using incremental quantile estimation as introduced by \cite{Tierney1983}. Although not currently part of the method in \texttt{ncvreg} at the time of this publication, this is an active area of implimentation. Incremental quantile estimation is an ongoing area of research, espeically in feilds like computer networking.

One concern here is that the additional approximation in incremental quantile estimation could result in larger errors, however, it is our belief that these errors will be small in relation to the approximations already a part of the method. Addiitonally, we believe that the errors introduced by incremental quantile estimation will quickly be offset by the ability to have obtained orders of magnitude more draws.

This of course opens up the large question of what method to use for incremental quantile estimation.

We are currently working on implimenting and assessing using the method introduced by \cite{Tierney1983}, but this is an area of ongoing interest for us.

Luckily, as far as problems go, this is a relatively benign use case. The primary obsticle that a method needs to be able to address is adhering to space requirements. Ideally, the storage space required should not grow with the number of draws or should at least grow at a much slower rate than the number of draws (i.e. no faster than $\log(\# of draws)$). In fact, per quantile tracked, the method presented by Tierney only requries 6 direct access storage points. As \cite{Tierney1983} notes, his method only has the requirement that F has "a bounded derivative that is continuous at $\xi$", where $\xi$ is the tracked quantile. The only point at which this could be violated for the distributions we are tracking would be if $\xi = 0$, since the derivative is not necessarily continuous at this point. However, since the distribution itself is continous at 0, $\xi = 0$ w.p. 0.

One drawback of using the method as introduced by Tierney is that is only addresses tracking of a single quantile at a time. Recent research has focused on joint quantile tracking. Future exploration into how such methods perform would undoubtably add robustness, however, given the use case here of tracking two quantiles in seperate tails, we accept this limitation with caution.

\section*{Acknowledgments}

\section*{Appendix}
